\documentclass[a4paper, 11pt]{article} % Font size (can be 10pt, 11pt or 12pt) and paper size (remove a4paper for US letter paper)

\usepackage[protrusion=true,expansion=true]{microtype} % Better typography
\usepackage{graphicx} % Required for including pictures
\usepackage{hyperref}
\usepackage{float}

\usepackage{mathpazo} % Use the Palatino font
\usepackage[T1]{fontenc} % Required for accented characters
\linespread{1.05} % Change line spacing here, Palatino benefits from a slight increase by default

\makeatletter
\renewcommand\@biblabel[1]{\textbf{#1.}} % Change the square brackets for each bibliography item from '[1]' to '1.'
\renewcommand{\@listI}{\itemsep=0pt} % Reduce the space between items in the itemize and enumerate environments and the bibliography

\renewcommand{\maketitle}{ % Customize the title - do not edit title and author name here, see the TITLE block below
\begin{flushright} % Right align
{\LARGE\@title} % Increase the font size of the title

\vspace{50pt} % Some vertical space between the title and author name

{\large\@author} % Author name
\\\@date % Date

\vspace{40pt} % Some vertical space between the author block and abstract
\end{flushright}
}

%----------------------------------------------------------------------------------------
%	TITLE
%----------------------------------------------------------------------------------------

\title{\textbf{Phylogeotool}\\ % Title
Installation Reference Manual} % Subtitle

\author{\textsc{Ewout Vanden Eynden, Pieter Libin, Kristof Theys, anderen, Guy Baele} % Author
\\{\textit{Rega Institute for Medical Research, KU Leuven}}} % Institution

\date{August 2016} % Date

%----------------------------------------------------------------------------------------

\begin{document}
\maketitle % Print the title section

\vspace{30pt} % Some vertical space between the abstract and first section

%------------------------------------------------
\tableofcontents
\newpage

\section{Installation}

\subsection{Using pre-built JAR files}
If you don't want to build the JARs/WAR from source, you can skip the following sections and proceed to section \ref{sec:jars}.
%PL: is a building from source really in its place in this document?
%GB: YES, it's an installation manual
%PL: I would just explain how to install the jars, and refer to a BUILD file that explains how to build the jars?
%GB: only referring to a build file requires even more knowledge than listing the steps here; we have to keep the fully detailed installation instructions, but can precede it by listing the main steps only.
%PL: With the BUILD file, I meant the build *documentation* file, that explains how to build the source code.
%PL: We could then keep the installation file about installing, and refer to this BUILD doc file from the installation manual.
%PL: But adding a section with details on how the build is performed, is also OK to me. 

\subsection{Prerequisites for building your own}

\subsubsection*{Java}
Just require Java 7 here?\\
%GB: edit this next time

Download and install the newest Java Development Kit (JDK) from \url{http://www.oracle.com/technetwork/java/javase/downloads/index.html}.
The current version of PhyloGeoTool was built using JDK 1.8.0\_31.
For example, on Ubuntu Linux, Oracle Java 8 can be installed as follows (although OpenJDK should work just fine):
\begin{verbatim} 
sudo add-apt-repository ppa:webupd8team/java
sudo apt-get update
sudo apt-get install oracle-java8-installer
\end{verbatim}

\subsubsection*{Tomcat}
%PL: this is not a prerequisite for building your own jars, this is confusing.
%PL: I would really focus on installing in this doc, not on Building.
%GB: as I said before, we can first present a short installation manual, but we must keep the detailed installation steps; I'm certainly NOT going to answer a bunch of e-mail asking for detailed instructions over and over again
%PL: I agree to that, and I never meant that we should omit this information, I believe we had a miscommuniction concerning what I referred to as BUILD file (this is how the build doc file was named in several projects I worked on) 
Download and install the latest Tomcat version from \url{http://tomcat.apache.org}.
For example:
\begin{verbatim}
sudo apt-get update
sudo apt-get install tomcat7
\end{verbatim}

\subsubsection*{Github}
Download the code from \url{https://github.com/rega-cev/phylogeotool/}. 
The project is currently still private. 
During the trial period you can send an email to \href{mailto:phylogeotool@kuleuven.be}  {phylogeotool@kuleuven.be} with your Github account name to get read rights on the project.
Git is readily available on most operating systems; if not, it can be installed as follows:
\begin{verbatim}
sudo apt-get install git-all
\end{verbatim}

\subsubsection*{Ant}
Download and install the newest Ant version from \url{http://ant.apache.org/} as we will use it to build our project.
The current version of PhyloGeoTool was built using Ant 1.9.4.
Ant can be installed as follows:
\begin{verbatim}
sudo apt-get install ant
\end{verbatim}

\subsubsection*{Phylogenetic tree}
A rooted binary phylogenetic tree in either Nexus or Newick format, which may or may not be time-stamped.

\subsubsection*{CSV file}
A comma-separated value (CSV) file, containing an ID column that contains IDs that correspond to the taxa names of the provided phylogenetic tree.
Remaining columns may contain additional information/annotation for the taxa in the phylogenetic tree, such as geographic location, virus genotype/subtype, patient attributes (age, ethnicity), \ldots
%PL: since the geographic location is interpreted to build up the map, I would think the application needs to know which column is providing info for this. Is this done by using a fixed name? If so mention it here.

\subsection{Building the project - Ubuntu Linux}

\subsubsection{Clone the git repository}

We here outline the various steps necessary for a successful build of the project.
The build process here is described as it was performed on an Ubuntu 14.04 LTS (Trusty Tahr) installation.
\begin{itemize}
\item Create an empty directory
\item {In that directory, clone the git repository: 
\begin{verbatim}
git clone https://github.com/rega-cev/phylogeotool/
cd phylogeotool
ant
\end{verbatim}
\item This build process generates a phylogeotool-01.war, DistanceMatrix.jar and PreRender.jar in the dist directory.
}
\end{itemize}

\subsubsection{DistanceMatrix.jar}
%GB: Is this mandatory? Why do we need to do this? Would be nice to explain here.
This jar file contains a tool to create a distance matrix based on the phylogenetic tree that will be used in the PhyloGeoTool.
DistanceMatrix.jar uses one input file, a phylogenetic tree to be used in the PhyloGeoTool, and will output the distance matrix that is derived from the phylogenetic tree. %GB: what is this phylo.tree? What format should it be in? Should it have a specific extension?
For example, the following command uses DistanceMatrix.jar to generate a distance matrix (to be stored in a file called distances.csv in the command below) from a previously reconstructed phylogenetic tree in Newick format: 
\begin{verbatim}
java -jar DistanceMatrix.jar tree.newick distances.csv
\end{verbatim}
%is the distance matrix a CSV file?

\subsubsection{PreRender.jar}
%GB: Does this step somehow require the distance matrix computed in the previous section?
%PL: yes
A major goal of the PhyloGeoTool is to cluster the tree index the taxa annotations, of which the computation takes a long time and its runtime depends on the size of the phylogenetic tree. 
However, this computation can be performed before the installation of the web-tool, using PreRender.jar, allowing the web-tool to render both clusters and annotations instantaneously.

%PL: This is all too technical, just require Java 7 for everything (I changed that too), and this needs no explenation.
%PreRender.jar is a multithreaded application, meaning that it performs best on any Java version > 7. 
%Older Java versions do not support our implementation of multithreading and should hence not be used.
PreRender.jar takes the following input files: 
%GB: are all of these mandatory or are some of these optional?
%PL: please change the command such that you only need to provide one directory, where all the subdirectories can be created (tree, clusters, csv, figtree).
%PL: This is much easier to use and to explain here.
%GB: I agree; will do during next writing session with Ewout
\begin{itemize}
\item /path/to/phylogenetic.tree: Link to the phylogenetic tree used in the PhyloGeoTool.
\item /path/to/csvFile: Link to the csv file that connects nodes in the tool to attributes. Note: The id in the csv file has to be the same as the id of the nodes in the tree.
\item /path/to/distance\_matrix: Link to the distance matrix that was generated from this tree. % PL: refer to the previous section
\item /path/to/folder\_xml\_tree: Link to the folder where this jar can write its resulting tree files.
\item /path/to/folder\_xml\_clusters: Link to the folder where this jar can write its resulting cluster files.
\item /path/to/folder\_xml\_csv: Link to the folder where this jar can write its resulting csv files.
\item /path/to/folder\_figtree: Link to the folder where this jar can write its resulting figtree representations.
\end{itemize}
%GB: do the folders above already have to exist? if not, will they be created automatically?
%PL: I don't think they are, so I think Ewout should change PreRender that it just accepts one directory, and if the sub-dirs are not there yet, creates them.
%PL: PS, if the dirs are there already, I would suggest giving a warning, rather than happily overwriting the content of the sub-dirs. 

For example, the following command uses PreRender.jar to perform all the necessary clustering steps (which can be time-consuming) so that the PhyloGeoTool doesn't have to perform these at run time: 
\begin{verbatim}
java -jar PreRender.jar phylogenetic.tree csvFile 
distance\_matrix folder\_xml\_tree folder\_xml\_clusters 
folder\_xml\_csv folder\_figtree
\end{verbatim}


\subsubsection{Create a config file}

In the /etc folder, create a directory `phylogeotool', which is the default location for the config file:
\begin{verbatim}
sudo mkdir phylogeotool
cd phylogeotool
touch global-conf.xml
\end{verbatim}

%GB: we should simply provide an example config file on github, which can easily be downloaded and adjusted
%GB: or perhaps a simple tool that consists of a series of file choosers to automatically generate the config file (AFTER WE SUBMIT)
%PL: I agree, the simple file should be linked to from the installation manual I think?

%PL: using a root account to run stuff with is generally considered bad practice, having this in our example is probably not a good idea? Use the user phylogeo?

%PL: Also, is it necessary to configure each of these directories (xml,tree,cluster,image,...) explicitely, why not 1 directory, where a set of sub-dirs is created in?
%PL: If you would like to keep all these separate options, I don't really mind, but I it should be possible with just 1 directory, this is much less work for the installer.
%PL: It will also make the reader of this documentation focus on the options that actually have an effect on the behaviour of the tool.

%PL: Also, does the notation ~/blabla works in Java?
%PL: If so, I would use it here, since it is less Mac specific. I guess a lot of possible installation will be done on GNU/Linu ... 
Edit the global-conf.xml file to correctly set up all the necessary paths, example content of global-conf.xml is shown here (assuming that all the data resides in subfolders of /Users/root/Documents/phylogeotool/):
\begin{verbatim}
<?xml version="1.0" encoding="UTF-8"?>
<phylogeotool-settings>
  <!-- xmlPath
     Directory containing the xml files and pre-rendered csv data,
     located in this project's xml/ directory.
   -->
  <property name="xmlPath">
    /Users/root/Documents/phylogeotool/xml
  </property>

  <!-- treePath
     Directory containing the xml representation of the full tree,
     located in this project's tree/ directory.
   -->
  <property name="treePath">
    /Users/root/Documents/phylogeotool/tree
  </property>
  
  <!-- clusterPath
   Directory containing the xml files and pre-rendered clusters, 
   located in this project's xml/ directory.
  -->
  <property name="clusterPath">
    /Users/root/Documents/phylogeotool/clusters
  </property>
  
  <!-- fullTreeImagesPath
   Directory containing the xml files and pre-rendered full tree 
   images, located in this projects xml/ directory.
  -->   
  <property name="fullTreeImagesPath">
    /Users/root/Documents/phylogeotool/treeview
  </property>

  <!-- metadataFile
   Pointer to the location of the csv file containing all the 
   information on the individual sequences.
  -->
  <property name="metadataFile">
   /Users/root/Documents/phylogeotool/information.csv
  </property>

  <!-- phyloTreeFile
   Pointer to the location of the phylogenetic tree file.
  -->
  <property name="phyloTreeFile">
   /Users/root/Documents/phylogeotool/phylogenetic.tree
  </property>

  <!-- alignmentFile
   Pointer to the location of the alignment file that was used 
   to build the phylogenetic tree used in this tool.
  -->
  <property name="alignmentFile">
    /Users/root/Documents/phylogeotool/alignment.fasta
  </property>
  
  <!-- logFile
   Pointer to the logfile file that was returned by the tree 
   building program to build the phylogenetic tree used in 
   this tool.
   %GB: this is not clear
  -->
  <property name="logFile">
    /Users/root/Documents/phylogeotool/logfile.log
  </property>

  <!-- scriptFolder
   Pointer to the location of the shell scripts that are used 
   to initiate PPlacer.
   %GB: where do these files come from? do they come with PPlacer?
   This folder contains init.sh and place.sh.
  -->
  <property name="scriptFolder">
    /Users/roott/Documents/pplacer_newTry
  </property>


  <!-- showNAData 
   Include Non Assigned (NA) data in the graphs.
  -->
  <property name="showNAData">
    false
  </property>
  
  <!-- visualizeGeography 
   Fill in the field that has to be visualized on the 
   map in the tool.
  -->
  <property name="visualizeGeography">
  	COUNTRY_OF_ORIGIN_EN
  </property>

  <!-- 
   Color codes used for the google chart
  -->
  <property name="colorCodes">
    <chart name="datalessregion">#FFFFFF</chart>
    <chart name="backgroundcolor">#FFFFFF</chart>    
    <chart name="colorAxis">#e9e9e9,red</chart>
  </property>
</phylogeotool-settings>
\end{verbatim}


\subsubsection{Start tomcat server and load phylogeotool}

Copy the phylogeotool-01.war to the webapps folder of Tomcat and start Tomcat:
\begin{verbatim}
sudo cp phylogeotool-01.war /var/lib/tomcat7/webapps/
cd /var/lib/tomcat7/
sudo service tomcat7 restart
\end{verbatim}
This enables browser access to a localhost version of the PhyloGeoTool.
Open a browser and enter the following URL: \url{http://localhost:8080/phylogeotool/PhyloGeoTool}.
The browser should show something similar to Figure \ref{fig:01}.


\begin{figure}[!htbp]
\includegraphics[scale=0.19]{images/defaultScreenshot.png}
\caption{Screenshot of the PhyloGeoTool application when the application is started for the first time.}
\label{fig:01} 
\end{figure}





\end{document}
