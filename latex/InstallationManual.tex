\documentclass[a4paper, 11pt]{article} % Font size (can be 10pt, 11pt or 12pt) and paper size (remove a4paper for US letter paper)

\usepackage[protrusion=true,expansion=true]{microtype} % Better typography
\usepackage{graphicx} % Required for including pictures
\usepackage{hyperref}
\usepackage{float}

\usepackage{mathpazo} % Use the Palatino font
\usepackage[T1]{fontenc} % Required for accented characters
\linespread{1.05} % Change line spacing here, Palatino benefits from a slight increase by default

\makeatletter
\renewcommand\@biblabel[1]{\textbf{#1.}} % Change the square brackets for each bibliography item from '[1]' to '1.'
\renewcommand{\@listI}{\itemsep=0pt} % Reduce the space between items in the itemize and enumerate environments and the bibliography

\renewcommand{\maketitle}{ % Customize the title - do not edit title and author name here, see the TITLE block below
\begin{flushright} % Right align
{\LARGE\@title} % Increase the font size of the title

\vspace{50pt} % Some vertical space between the title and author name

{\large\@author} % Author name
\\\@date % Date

\vspace{40pt} % Some vertical space between the author block and abstract
\end{flushright}
}

%----------------------------------------------------------------------------------------
%	TITLE
%----------------------------------------------------------------------------------------

\title{\textbf{Phylogeotool}\\ % Title
Installation Reference Manual} % Subtitle

\author{\textsc{Ewout Vanden Eynden, Pieter Libin, Kristof Theys, anderen, Guy Baele} % Author
\\{\textit{Rega Institute for Medical Research, KU Leuven}}} % Institution

\date{August 2016} % Date

%----------------------------------------------------------------------------------------

\begin{document}
\maketitle % Print the title section

\vspace{30pt} % Some vertical space between the abstract and first section

%------------------------------------------------
\tableofcontents
\newpage

\section{Installation}

\subsection{Prerequisites}

\subsubsection*{Java}
Download and install the newest Java Development Kit (JDK) from \url{http://www.oracle.com/technetwork/java/javase/downloads/index.html}.
The current version of PhyloGeoTool was built using JDK 1.8.0\_31.

\subsubsection*{Tomcat}
Download and install the latest Tomcat version from \url{http://tomcat.apache.org}.

\subsubsection*{Github}
Download the code from \url{https://github.com/rega-cev/phylogeotool/}. 
The project is currently still private. 
During the trial period you can send an email to \href{mailto:phylogeotool@kuleuven.be}  {phylogeotool@kuleuven.be} with your Github account name to get read rights on the project.

\subsubsection*{Ant}
Download and install the newest Ant version from \url{http://ant.apache.org/} as we will use it to build our project.
The current version of PhyloGeoTool was built using Ant 1.9.4

\subsubsection*{Phylogenetic tree} %GB: we also need a phylogenetic tree to deploy the application, right? Or to at least perform the distance matrix calculations and such? So what format should this tree be in? Can it be a time-stamped tree, i.e. with heterochronous taxa? Can it be rooted or unrooted?



\subsection{Building the project}

\subsubsection{Ubuntu Linux}

We here outline the various steps necessary for a successful build of the project.
The build process here is described as it was performed on an Ubuntu 14.04 LTS (Trusty Tahr) installation.
\begin{itemize}
\item Go the the following directory:
\item In that directory, you should see the following files:
\item \ldots
\end{itemize}

\section{Data preparation}
After the project has been built successfully, different jar files can be found in the dist folder. 
Their function is explained in the sections below.

\subsection{DistanceMatrix.jar}
%GB: Is this mandatory? Why do we need to do this? Would be nice to explain here.
This jar file contains a tool to create a distance matrix based on the phylogenetic tree that will be used in the PhyloGeoTool.
DistanceMatrix.jar takes the following input values:
\begin{itemize}
\item /path/to/phylogenetic.tree: Link to the phylogenetic tree to be used in the PhyloGeoTool. %GB: what is this phylo.tree? What format should it be in? Should it have a specific extension?
\item /path/to/distance\_matrix: Link to the location where the distance matrix can be written. Make sure you have write privileges in the provided directory.
\end{itemize}
For example, the following command uses DistanceMatrix.jar to generate a distance matrix from a previously reconstructed phylogenetic tree: %GB: add example command


\subsection{PreRender.jar}
%GB: Does this step somehow require the distance matrix computed in the previous section?
Before any data can be shown to the user, a lot of calculations are done at the back end. 
To speed up the visualization process, most of those calculations can be done up front.

PreRender.jar is a multithreaded application, meaning that it performs best on any Java version > 7. 
Older Java versions do not support our implementation of multithreading and should hence not be used.
PreRender.jar takes the following input files: %GB: are all of these mandatory or are some of these optional?
\begin{itemize}
\item /path/to.phylogenetic.tree: Link to the phylogenetic tree used in the PhyloGeoTool.
\item /path/to/csvFile: Link to the csv file that connects nodes in the tool to attributes. Note: The id in the csv file has to be the same as the id of the nodes in the tree.
\item /path/to/distance\_matrix: Link to the distance matrix that was generated from this tree.
\item /path/to/folder\_xml\_tree: Link to the folder where this jar can write its resulting tree files.
\item /path/to/folder\_xml\_clusters: Link to the folder where this jar can write its resulting cluster files.
\item /path/to/folder\_xml\_csv: Link to the folder where this jar can write its resulting csv files.
\item /path/to/folder\_figtree: Link to the folder where this jar can write its resulting figtree representations.
\end{itemize}

For example, the following command uses PreRender.jar to \ldots: %GB: add example command
%GB: and now what? Are we done here? Shall we show what the user should see now and what he/she can do next?


\end{document}
